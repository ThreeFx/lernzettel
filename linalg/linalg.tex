\documentclass[a4paper,12pt]{article}

\usepackage[T1]{fontenc}
\usepackage[utf8]{inputenc}

\usepackage[left=2cm,right=3cm,top=2cm,bottom=2.5cm]{geometry}

\usepackage{parskip}

\usepackage{tikz-cd}

\usepackage{amsmath, amsthm, amssymb, amsfonts}
\usepackage{mathtools}

\usepackage{mdframed}

\usepackage{xcolor}

\newcommand{\uniti}{i}
\newcommand{\defas}{\coloneqq}
\newcommand{\fto}{\rightarrow}

\renewcommand{\Re}[1]{\operatorname{Re}{#1}}
\renewcommand{\Im}[1]{\operatorname{Im}{#1}}
\newcommand{\conj}[1]{\bar{#1}}

\newcommand{\E}{\mathbb{E}}
\newcommand{\R}{\mathbb{R}}
\newcommand{\C}{\mathbb{C}}

% Shortcuts for lin. alg
\newcommand{\I}{\mat{I}}
\newcommand{\A}{\mat{A}}
\newcommand{\B}{\mat{B}}
\newcommand{\T}{\mat{T}}

\newcommand{\rank}{\operatorname{rank}}
\newcommand{\trace}{\operatorname{trace}}
\newcommand{\diag}{\operatorname{diag}}

\newcommand{\Er}{E_{\lambda}}
\newcommand{\charp}[1]{\chi_{#1}}

\newcommand{\mat}[1]{\mathbf{#1}}
\newcommand{\inv}[1]{{#1}^{-1}}

\let\oldxi\xi
\renewcommand{\xi}{\mathbf{\oldxi}}

\let\oldeta\eta
\renewcommand{\eta}{\mathbf{\oldeta}}

\setcounter{section}{-1}

\theoremstyle{plain}
\newtheorem{defn}{Definition}[section]
\newtheorem{lemma}{Lemma}[section]
\newtheorem{theorem}[lemma]{Theorem}
\newtheorem{corollary}[lemma]{Corollary}

\newtheorem{alg}{Algorithm}[section]
%\newmdtheoremenv [backgroundcolor=gray, %
%innertopmargin=0pt, %
%splittopskip=\topskip, %
%skipbelow=6pt,%
%skipabove=6pt,%
%topline=false,bottomline=false,leftline=false,rightline=false]{alg}{Algorithm}[section]


\numberwithin{equation}{section}

\begin{document}

\author{Ben Fiedler}
\title{Linear Algebra \\
       \large{Özlem Imamoglu, Olga Sorkine-Hornung}}

\maketitle
\tableofcontents

\newpage

\section{Definitions}

\subsection{Complex numbers}

The complex numbers are an extension of the real numbers which posess the ability to
solve equations of the form:

$$ (x + 5) ^ 2 = -81 $$

\subsubsection{Imaginary unit}

In $\R$, no solutions exist for the equation:

\begin{equation}
    x^2 = {-1}
\end{equation}

To solve these, a number $\uniti$ is introduced with the following property:

\begin{equation}
    \uniti^2 = {-1}
\end{equation}

$\uniti$ is called the imaginary unit, as it behaves similar to $1$ with respect to
the imaginary numbers. Complex numbers are a combination of a real and an imaginary number:

\begin{equation}
    z \defas (x, y) \defas z = x + \uniti y \quad \text{with} \quad x, y \in \R
\end{equation}

$x$ is called the real part of $z$, denoted as $\Re{z}$ and $y$ is called the
imaginary part of $z$, denoted as $\Im{z}$:

$$ \Re{z} \defas x \quad \text{and} \quad \Im{z} \defas y $$

\subsubsection{Operations}

The complex numbers form a \textit{field}. For two complex numbers
$ z = x + \uniti y $ and $ w = u + \uniti v $ we can define
addition and multiplication as follows:

\begin{defn} Addition
    $$ z + w \defas (x + u) + \uniti (y + v) $$
\end{defn}
\begin{defn} Multiplication
    $$ z \times w \defas (xu - yw) + \uniti (xv + uy) $$
\end{defn}

Furthermore, we can define an additional operation, called complex conjugate.

\begin{defn}
    For every complex number $z = x + \uniti y$,
    we can define it's complex conjugate $\conj{z}$ as follows:
    $$ \conj{z} \defas x - \uniti y $$
\end{defn}

\section{Linear Equations}

\section{Matrices and Vectors in $\R^{n}$ and $\mathbb{C}^{n}$}

\subsection{LU decomposition}

\section{Vector spaces}

\section{Linear transformations}

\section{Scalar product}

\section{Least squares method and QR decomposition}

\section{Determinants}

\section{Eigenvectors and Eigenvalues}

\subsection{Definition}

\begin{defn}
\label{eigenvalue}
    Let $F \colon V \rightarrow V$ be a linear mapping on the vector space $V$.
    A value $\lambda$ is an Eigenvalue if there exists an Eigenvector $x \neq 0$
    such that:

    \begin{equation}
        F(x) = \lambda x
    \end{equation}
\end{defn}

An Eigenvalue $\lambda$ defines an Eigenspace $\Er$, which is a subspace
of $V$:

\begin{equation}
    \Er \defas \{ v \in V \mid F(v) = \lambda v \}
\end{equation}

The set of all Eigenvalues of $F$ is denoted as spectrum of $F$.

Since every matrix $\A$ defines a linear mapping $F$ and vice versa,
matrices also have Eigenvalues $\lambda$ defined as follows: $\lambda$ is
an Eigenvalue of $\A$ iff there exists a vector $\xi$ such that:

\begin{equation}
    \A \xi = \xi \lambda
\end{equation}

\begin{lemma}
\label{lem:equiv-map-mat}
    Let $F \colon V \rightarrow V$ be a linear mapping,
    $\kappa_{V} \colon V \rightarrow \E^{n}$ a coordinate mapping for an
    arbitrary basis and $\A = \kappa_{V} F \inv{\kappa_{V}}$ the matrix
    corresponding to $F$. Then we have:

    \begin{equation*}
        \begin{gathered}
            \lambda \enspace \text{EVal of} \enspace F
                \iff \lambda \enspace \text{EVal of} \enspace \A \\
            x \enspace \text{EVec of} \enspace F
                \iff x \enspace \text{EVec of} \enspace \A
        \end{gathered}
    \end{equation*}
\end{lemma}

\begin{proof}
    From $F x = \lambda x$ follows $\kappa_{V}(F x) = \kappa_{V}(\lambda x) =
    \lambda \kappa_{V}(x) = \lambda \xi$. From the definition of $\kappa_{V}$ and
    $\A$, we have $\A \xi = \kappa_{V}(F x)$, so $\A \xi = \lambda \xi$.
    From $\inv{\kappa_{V}}$ the opposite follows.
\end{proof}

An Eigenvector $v$ for an Eigenvalue $\lambda$ is not unique, as we can see that:

\begin{equation*}
    F v = \lambda v \quad \implies \quad F (\alpha v) = \lambda (\alpha v).
\end{equation*}

holds due to the linearity of $F$. From this follows that the Eigenspace $\Er$
of an Eigenvalue $\lambda$ is at least 1-dimensional.

The Eigenraum $\Er$ is the set of vectors $v$ for which:

\begin{equation}
    (F - \lambda I) v = 0
\end{equation}

\begin{lemma}
\label{lem:ev-kern}
    $\lambda$ is an Eigenvalue of $F \colon V \rightarrow V$ iff $\ker (F - \lambda \I)$
    contains more than the null vector. The Eigenspace $\Er$ is a
    subspace of $V$ greater than the null space defined as:
    \begin{equation}
        \Er = \ker (F - \lambda \I)
    \end{equation}
\end{lemma}

\begin{proof}
    Assume $\lambda$ is an Eigenvalue of $F$. Then we follow:
    \begin{align*}
        F v = \lambda v \ &\iff \ F v - \lambda v = 0 \\
        &\iff \ (F - \lambda \I) v = 0 \\
        &\iff \ v \in \ker (F - \lambda \I).
    \end{align*}
    The reverse follows in the same fashion.
\end{proof}

\begin{defn}
\label{geo-mult}
    The \textbf{geometric multiplicity} of an Eigenvalue $\lambda$ is equal to
    the dimension of $\Er$.
\end{defn}

From Lemma \ref{lem:ev-kern} follows that:

\begin{corollary}
\label{er-mat}
    $\lambda$ is an Eigenvalue of $\A \in \E^{n \times n}$, iff $\A - \lambda \I$ is singular.
    The Eigenspace $\Er$ of an Eigenvalue $\lambda$ of $\A$ is a subspace,
    different from the null space, which is:

    \begin{equation}
        \Er = \ker (A - \lambda \I)
    \end{equation}

    The \textbf{geometric multiplicity} of $\lambda$ is therefore:

    \begin{equation}
    \label{geo-mult-mat}
        \dim \Er = \dim \ker (\A - \lambda \I) = n - \rank (\A - \lambda \I)
    \end{equation}
\end{corollary}

\begin{proof}
    The second equality in \ref{geo-mult-mat} follows from the dimension formula.
    The rest follows from Lemma \ref{lem:ev-kern}.
\end{proof}

How can we find Eigenvalues? We have to choose $\lambda$ so that $\A - \lambda \I$
is singular, which happens iff $\det (\A - \lambda \I) = 0$. The formula of
Sarrus gives us:

\begin{equation}
    \begin{split}
        \det \A &=
        \begin{vmatrix}
            a_{11} - \lambda & a_{12} & \dots  & a_{1n} \\
            a_{21} & a_{22} - \lambda & \dots  & a_{2n} \\
            \vdots & \vdots           & \ddots & \vdots \\
            a_{n1} & a_{n2}           & \dots  & a_{nn} - \lambda \\
        \end{vmatrix}
        \\[0.3cm]
        &= (-\lambda)^{n} + (a_{11} + a_{22} + \dots + a_{nn}) (-\lambda)^{n-1} + \dots + \det \A
        \\
        &\eqqcolon \charp{\A}(\lambda)
    \end{split}
\end{equation}

$\charp{\A}(\lambda)$ is a polynomial in $\lambda$ of degreen $n$.

\begin{defn}
    The polynomial
    \begin{equation}
        \charp{\A}(\lambda) \coloneqq \det (\A - \lambda \I)
    \end{equation}
    is called \textbf{characteristic polynomial} of $\A \in \E^{n \times n}$ and
    the equation
    \begin{equation}
        \charp{\A}(\lambda) = 0
    \end{equation}
    is called \textbf{characteristic equation}.
\end{defn}

The sum of the diagonal elements is called \textbf{trace}.

\begin{equation}
    \trace \A = a_{11} + a_{22} + \dots + a_{nn}
\end{equation}

\begin{lemma}
    \label{lem:charp-form}
    The characteristic polynomial $\charp{\A}$ has the form
    \begin{equation}
        \begin{split}
            \charp{\A}(\lambda) &= \det (\A - \lambda \I) \\
            &= (-\lambda)^{n} + \trace \A (-\lambda)^{n-1} + \dots + \det \A
        \end{split}
    \end{equation}
\end{lemma}

\begin{proof}
    The proof is obvious by filling in the definitions.
\end{proof}

\begin{theorem}
    \label{the:ev-is-root}
    $\lambda \in \E$ is an Eigenvalue of $\A$ iff $\lambda$ is a root of the
    characteristic polynomial $\charp{\A}$ or a solution to the characteristic
    equation, respectively.
\end{theorem}

\begin{proof}
    Again, the proof is obvious by filling in the definitions.
\end{proof}

By the fundamental theorem of algebra, every polynomial of degree $n$ has
exactly $n$ roots in $\C$. So, $\charp{\A}$, which is of degree $n$ and has $(-1)^{n}$ as its highest
coefficient, can be written as:

\begin{equation*}
    \charp{\A} = (\lambda_{1} - \lambda)(\lambda_{2} - \lambda)\dots(\lambda_{n} - \lambda),
\end{equation*}

where $\lambda_{i} \in \C, \ i \in \{1, \dots, n\}$

A polynomial over $\R$ can have complex roots, since $\R \subset \C$. Because of this,
we assume $\E = \C$ for the rest of this section because otherwise, for a matrix
$\A \in \R^{n \times n}$, there may exist an Eigenvalue $\lambda \not\in \R$ and a Eigenvector
$v \in \R^{n}$ such that $\A v \in \R^{n}$ but $v \lambda \not\in \R^{n}$.

The geometric multiplicity of an Eigenvalue $\lambda$ may or may not be equal to its
multiplicity as root of $\charp{\A}$. Therefore we define another property of
Eigenvalues as follows:

\begin{defn}
    The \textbf{algebraic multiplicity} of an Eigenvalue is equal to its multiplicity
    as a root of the characteristic polynomial $\charp{\A}$
\end{defn}

The basis vector of the Eigenspaces of a matrix $\A$ can be determined as follows: \\

\begin{alg}
    For a matrix $\A \in \C^{n \times n}$ we can determine a basis for its
    Eigenspace as follows:

    \begin{enumerate}
        \item{Determine the characteristic polynomial $\charp{\A}$.}
        \item{Compute the roots $\lambda_{1} \dots \lambda_{n}$ of $\charp{\A}$.}
        \item{For every Eigenvalue $\lambda_{k}$: Determine the basis of
            $\ker (\A - \lambda_{k} \I)$ using the Gauss-Jordan elimination.}
    \end{enumerate}
\end{alg}

Furthermore, it holds that:

\begin{lemma}
    For any matrix $\A \in \C^{n \times n}$, $\A$ is singular iff 0 is an
    Eigenvalue of $\A$:

    $$ \A \quad \text{singular} \quad \iff \quad 0 \in \sigma(\A) $$
\end{lemma}

\begin{proof}
    $\A$ is singular iff $\A x = 0$ has nontrivial solutions, which implies that
    $0$ must be an Eigenvalue of $\A$.
\end{proof}

Alternatively, if $\A$ is singular, then $\det \A = 0$. But then the characteristic
polynomial $\charp{\A}$ is of the form $(-\lambda)^{n} + \trace \A (-\lambda)^{n - 1} + \dots + 0$
which in turn implies that one can factor out $-\lambda$:

\begin{align*}
     & (-\lambda)^{n} + \trace \A (-\lambda)^{n - 1} + \dots + 0 \\
    =& (-\lambda) ((-\lambda)^{n-1} + \trace \A (-\lambda)^{n-2} + \dots) \\
    =& (0 - \lambda) ((-\lambda)^{n-1} + \trace \A (-\lambda)^{n-2} + \dots) \\
\end{align*}

This shows that zero is a root of $\charp{\A}$, which in turn implies that 0 is an
Eigenvalue of $\A$.

\subsection{Similarity transformation}

We now look at linear maps $F \colon V \rightarrow V$, $x \mapsto F x$ of a vector
space $V$ with Dimension $n$ and the transformation matrices $\A$ and $\B$ for the
bases $\{b_1, b_2, \dots, b_n\}$ and $\{b'_1, b'_2, \dots ,b'_n\}$.

\[
    \begin{tikzcd}[row sep=large, column sep=large]
    x \in V \arrow{r}{F} \arrow{d}{\kappa_V} & y \in V \arrow{d}{\kappa_V} \\
    \xi \in \E^n \arrow{r}{\A} \arrow{u}{\inv{\kappa_V}} \arrow{d}{\inv{\mat{T}}} & \eta \in \E^n \arrow{u}{\inv{\kappa_V}} \arrow{d}{\inv{\mat{T}}} \\
    \xi' \in \E^n \arrow{r}{\B} \arrow{u}{\mat{T}} & \eta' \in \E^n \arrow{u}{\mat{T}}
\end{tikzcd}
\]

Remember that, according to the diagram:

\begin{equation}
    \B = \inv{\mat{T}} \A \mat{T}, \quad \A = \mat{T} \B \inv{\mat{T}}
\end{equation}

which states that $\A$ and $\B$ are similar. The transformation
$\A \mapsto \B = \inv{\mat{T}} \A \mat{T}$ is called \textbf{similarity transformation}.

We ask us how the transformation matrix can be simplified by choosing an appropriate basis.
Formulated differently: How can the matrix $\A$ be simplified by a similarity transformation
$\A \mapsto \B = \inv{\mat{T}} \A \mat{T}$. This question is closely related to
the theory of Eigenvalues.

\begin{theorem}
    Similar matrices have the same characteristic polynomial $\charp{\A}$; They
    also have the same determinant, the same trace and the same Eigenvalues.

    Both the algebraic and geometric multiplicities of all Eigenvalues are the same.
\end{theorem}

\begin{proof}
    Let $\B = \inv{\mat{T}} \A \mat{T}$. Then we know the following about $\charp{\B}$:
    \begin{align*}
        \charp{\B} &= \det(\B - \lambda \I) \\
        &= \det(\inv{\T} \A \T - \lambda \I) \\
        &= \det((\inv{\T} \A - \lambda \inv{\T}) \T) \\
        &= \det(\inv{\T} (\A - \lambda \I) \T) \\
        &= \det(\inv{\T}) \det(\A - \lambda \I) \det(\T) \\
        &= \det(\A - \lambda \I) \\
        &= \charp{\A}
    \end{align*}

    From lemma \ref{lem:charp-form} and theorem \ref{the:ev-is-root} we can see that
    the determinants, traces and eigenvalues of $\A$ and $\B$ and their algebraic multiplicities match.

    We can see that their geometric multiplicities match by lemma \ref{lem:equiv-map-mat}.

    \begin{align*}
        \underbrace{\inv{\T} \A \T}_{= \B} \underbrace{\inv{\T} v}_{\eqqcolon w} &= \underbrace{\inv{\T} v}_{\eqqcolon w} \lambda \\
        \underbrace{\T \B \inv{\T}}_{= \A} \underbrace{\T w}_{\eqqcolon v} &= \underbrace{\T w}_{\eqqcolon v} \lambda
    \end{align*}

    We see that:
    \begin{equation}
        \Er(\B) = \inv{\T} \Er(\A), \quad \Er(\A) = \T \Er(\B)
    \end{equation}

    Since $\T$ is regular, we follow that $\dim \Er(A) = \dim \Er(\B)$.
\end{proof}

If a vector space $V$ has a basis of eigenvectors, then the transformation matrix
$\A$ is diagonal.

\begin{lemma}
    A transformation matrix to a linear map $F \colon V \rightarrow V$ is diagonal
    iff the Basis of $V$ contains only eigenvectors.
\end{lemma}

\begin{proof}
    If $v_1, v_2, \dots, v_n$ are eigenvectors which form a basis of $V$ with
    $\lambda_1, \lambda_2, \dots, \lambda_n$ as the corresponding eigenvalues, then
    $F v_l = v_l \lambda_l$ has the coordinate vector
    \[
        (\ 0\ \dots\ 0\ \lambda_l\ 0\ \dots\ 0\ )^\intercal
    \]
which shows that the transformation matrix $\A = \mat{\Lambda} = \diag(\lambda_1, \lambda_2, \dots, \lambda_n)$.
\end{proof}

\begin{defn}
    A basis of eigenvectors of $F$ is calles eigenbasis.
\end{defn}

\end{document}
