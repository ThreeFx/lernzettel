\documentclass[a4paper,12pt]{article}

\usepackage[T1]{fontenc}
\usepackage[utf8]{inputenc}

\usepackage{parskip}

\usepackage{amsmath, amsthm, amssymb, amsfonts}
\usepackage{mathtools}

\newcommand{\uniti}{i}
\newcommand{\defas}{\coloneqq}
\newcommand{\fto}{\rightarrow}

\renewcommand{\Re}[1]{\operatorname{Re}{#1}}
\renewcommand{\Im}[1]{\operatorname{Im}{#1}}
\newcommand{\conj}[1]{\bar{#1}}

\newcommand{\E}{\mathbb{E}}
\newcommand{\R}{\mathbb{R}}
\newcommand{\C}{\mathbb{C}}

% Shortcuts for lin. alg
\newcommand{\I}{\mat{I}}
\newcommand{\A}{\mat{A}}

\newcommand{\rank}[1]{\operatorname{rank} #1}

\newcommand{\Er}{E_{\lambda}}

\newcommand{\mat}[1]{\mathbf{#1}}
\newcommand{\inv}[1]{{#1}^{-1}}

%\newcommand{\ker}{\text{ker}}

\let\oldxi\xi
\renewcommand{\xi}{\mathbb{\oldxi}}

\setcounter{section}{-1}

\theoremstyle{plain}
\newtheorem{defn}{Definition}[section]
\newtheorem{lemma}{Lemma}[section]
\newtheorem{theorem}{Threorem}[lemma]
\newtheorem{corollary}{Corollary}[lemma]


\numberwithin{equation}{section}

\begin{document}

\author{Ben Fiedler}
\title{Linear Algebra \\
       \large{Özlem Imamoglu, Olga Sorkine-Hornung}}
\date{ }

\maketitle
\tableofcontents

\newpage

\section{Definitions}

\subsection{Complex numbers}

The complex numbers are an extension of the real numbers which posess the ability to
solve equations of the form:

$$ (x + 5) ^ 2 = -81 $$

\subsubsection{Imaginary unit}

In $\R$, no solutions exist for the equation:

\begin{equation}
    x^2 = {-1}
\end{equation}

To solve these, a number $\uniti$ is introduced with the following property:

\begin{equation}
    \uniti^2 = {-1}
\end{equation}

$\uniti$ is called the imaginary unit, as it behaves similar to $1$ with respect to
the imaginary numbers. Complex numbers are a combination of a real and an imaginary number:

\begin{equation}
    z \defas (x, y) \defas z = x + \uniti y \quad \text{with} \quad x, y \in \R
\end{equation}

$x$ is called the real part of $z$, denoted as $\Re{z}$ and $y$ is called the
imaginary part of $z$, denoted as $\Im{z}$:

$$ \Re{z} \defas x \quad \text{and} \quad \Im{z} \defas y $$

\subsubsection{Operations}

The complex numbers form a \textit{field}. For two complex numbers
$ z = x + \uniti y $ and $ w = u + \uniti v $ we can define
addition and multiplication as follows:

\begin{defn} Addition
    $$ z + w \defas (x + u) + \uniti (y + v) $$
\end{defn}
\begin{defn} Multiplication
    $$ z \times w \defas (xu - yw) + \uniti (xv + uy) $$
\end{defn}

Furthermore, we can define an additional operation, called complex conjugate.

\begin{defn}
    For every complex number $z = x + \uniti y$,
    we can define it's complex conjugate $\conj{z}$ as follows:
    $$ \conj{z} \defas x - \uniti y $$
\end{defn}

\section{Linear Equations}

\section{Matrices and Vectors in $\R^{n}$ and $\mathbb{C}^{n}$}

\subsection{LU decomposition}

\section{Vector spaces}

\section{Linear transformations}

\section{Scalar product}

\section{Least squares method and QR decomposition}

\section{Determinants}

\section{Eigenvectors and Eigenvalues}

\subsection{Definition}

\begin{defn}
\label{eigenvalue}
    Let $F \colon V \rightarrow V$ be a linear mapping on the vector space $V$.
    A value $\lambda$ is an Eigenvalue if there exists an Eigenvector $x \neq 0$
    such that:

    \begin{equation}
        F(x) = \lambda x
    \end{equation}
\end{defn}

An Eigenvalue $\lambda$ defines an Eigenspace $\Er$, which is a subspace
of $V$:

\begin{equation}
    \Er \defas \{ v \in V \mid F(v) = \lambda v \}
\end{equation}

The set of all Eigenvalues of $F$ is denoted as spectrum of $F$.

Since every matrix $\A$ defines a linear mapping $F$ and vice versa,
matrices also have Eigenvalues $\lambda$ defined as follows: $\lambda$ is
an Eigenvalue of $\A$ iff there exists a vector $\xi$ such that:

\begin{equation}
    \A \xi = \xi \lambda
\end{equation}

\begin{lemma}
\label{equiv-map-mat}
    Let $F \colon V \rightarrow V$ be a linear mapping,
    $\kappa_{V} \colon V \rightarrow \E^{n}$ a coordinate mapping for an
    arbitrary basis and $\A = \kappa_{V} F \inv{\kappa_{V}}$ the matrix
    corresponding to $F$. Then we have:

    \begin{equation*}
        \begin{gathered}
            \lambda \enspace \text{EVal of} \enspace F
                \iff \lambda \enspace \text{EVal of} \enspace \A \\
            x \enspace \text{EVec of} \enspace F
                \iff x \enspace \text{EVec of} \enspace \A
        \end{gathered}
    \end{equation*}
\end{lemma}

\begin{proof}
    From $F x = \lambda x$ follows $\kappa_{V}(F x) = \kappa_{V}(\lambda x) =
    \lambda \kappa_{V}(x) = \lambda \xi$. From the definition of $\kappa_{V}$ and
    $\A$, we have $\A \xi = \kappa_{V}(F x)$, so $\A \xi = \lambda \xi$.
    From $\inv{\kappa_{V}}$ the opposite follows.
\end{proof}

An Eigenvector $v$ for an Eigenvalue $\lambda$ is not unique, as we can see that:

\begin{equation*}
    F v = \lambda v \quad \implies \quad F (\alpha v) = \lambda (\alpha v).
\end{equation*}

holds due to the linearity of $F$. From this follows that the Eigenspace $\Er$
of an Eigenvalue $\lambda$ is at least 1-dimensional.

The Eigenraum $\Er$ is the set of vectors $v$ for which:

\begin{equation}
    (F - \lambda I) v = 0
\end{equation}

\begin{lemma}
\label{lem:ev-kern}
    $\lambda$ is an Eigenvalue of $F \colon V \rightarrow V$ iff $\ker (F - \lambda \I)$
    contains more than the null vector. The Eigenspace $\Er$ is a
    subspace of $V$ greater than the null space defined as:
    \begin{equation}
        \Er = \ker (F - \lambda \I)
    \end{equation}
\end{lemma}

\begin{proof}
    Assume $\lambda$ is an Eigenvalue of $F$. Then we follow:
    \begin{align*}
        F v = \lambda v \ &\iff \ F v - \lambda v = 0 \\
        &\iff \ (F - \lambda \I) v = 0 \\
        &\iff \ v \in \ker (F - \lambda \I).
    \end{align*}
    The reverse follows in the same fashion.
\end{proof}

\begin{defn}
\label{geo-mult}
    The \textbf{geometric multiplicity} of an Eigenvalue $\lambda$ is equal to
    the dimension of $\Er$.
\end{defn}

From Lemma \ref{lem:ev-kern} follows that:

\begin{corollary}
\label{er-mat}
    $\lambda$ is an Eigenvalue of $\A \in \E^{n \times n}$, iff $\A - \lambda \I$ is singular.
    The Eigenspace $\Er$ of an Eigenvalue $\lambda$ of $\A$ is a subspace,
    different from the null space, which is:

    \begin{equation}
        \Er = \ker (A - \lambda \I)
    \end{equation}

    The \textbf{geometric multiplicity} of $\lambda$ is therefore:

    \begin{equation}
        \dim \Er = \dim \ker (\A - \lambda \I) = n - \rank (\A - \lambda \I)
    \end{equation}
\end{corollary}

















\end{document}
