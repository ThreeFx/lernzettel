\documentclass[a4paper,12pt]{article}

\usepackage[T1]{fontenc}
\usepackage[utf8]{inputenc}

\usepackage{parskip}

\usepackage{amsmath, amsthm, amssymb, amsfonts}
\usepackage{colonequals}

\newcommand{\uniti}{i}
% also called \logeq "logically equivalent to"
\newcommand{\defas}{\ratio\Leftrightarrow}

\renewcommand{\Re}[1]{\operatorname{Re}{#1}}
\renewcommand{\Im}[1]{\operatorname{Im}{#1}}
\newcommand{\conj}[1]{\bar{#1}}

\setcounter{section}{-1}

\theoremstyle{definition}
\newtheorem{defn}{Definition}[section]


\begin{document}

\author{Ben Fiedler}
\title{Linear Algebra \vspace{.5cm}
       \large{Özlem Imamoglu, Olga Sorkine-Hornung}}
\date{ }

\maketitle
\tableofcontents

\newpage

\section{Definitions}

\subsection{Complex numbers}

The complex numbers are an extension of the real numbers which posess the ability to
solve equations of the form:

$$ (x + 5) ^ 2 = -81 $$

\subsubsection{Imaginary unit}

In $\mathbb{R}$, no solutions exist for the equation:

\begin{equation}
    x^2 = {-1}
\end{equation}

To solve these, a number $\uniti$ is introduced with the following property:

\begin{equation}
    \uniti^2 = {-1}
\end{equation}

$\uniti$ is called the imaginary unit, as it behaves similar to $1$ with respect to
the imaginary numbers. Complex numbers are a combination of a real and an imaginary number:

\begin{equation}
    z = (x, y) \defas z = x + \uniti y \quad \text{with} \quad x, y \in \mathbb{R}
\end{equation}

$x$ is called the real part of $z$, denoted as $\Re{z}$ and $y$ is called the
imaginary part of $z$, denoted as $\Im{z}$:

$$ \Re{z} \defas x \quad \text{and} \quad \Im{z} \defas y $$

\subsubsection{Operations}

The complex numbers form a \textit{field}. For two complex numbers
$ z = x + \uniti y $ and $ w = u + \uniti v $ we can define
addition and multiplication as follows:

\begin{defn} Addition
    $$ z + w \defas (x + u) + \uniti (y + v) $$
\end{defn}
\begin{defn} Multiplication
    $$ z \times w \defas (xu - yw) + \uniti (xv + uy) $$
\end{defn}

Furthermore, we can define an additional operation, called complex conjugate.

\begin{defn}
    For every complex number $z = x + \uniti y$,
    we can define it's complex conjugate $\conj{z}$ as follows:
    $$ \conj{z} \defas x - \uniti y $$
\end{defn}

\section{Linear Equations}

\section{Matrices and Vectors in $\mathbb{R}^{n}$ and $\mathbb{C}^{n}$}

\subsection{LU decomposition}

\section{Vector spaces}

\section{Linear transformations}

\section{Scalar product}

\section{Least squares method and QR decomposition}

\end{document}
